% Options for packages loaded elsewhere
\PassOptionsToPackage{unicode}{hyperref}
\PassOptionsToPackage{hyphens}{url}
%
\documentclass[
  ignorenonframetext,
]{beamer}
\usepackage{pgfpages}
\setbeamertemplate{caption}[numbered]
\setbeamertemplate{caption label separator}{: }
\setbeamercolor{caption name}{fg=normal text.fg}
\beamertemplatenavigationsymbolsempty
% Prevent slide breaks in the middle of a paragraph
\widowpenalties 1 10000
\raggedbottom
\setbeamertemplate{part page}{
  \centering
  \begin{beamercolorbox}[sep=16pt,center]{part title}
    \usebeamerfont{part title}\insertpart\par
  \end{beamercolorbox}
}
\setbeamertemplate{section page}{
  \centering
  \begin{beamercolorbox}[sep=12pt,center]{part title}
    \usebeamerfont{section title}\insertsection\par
  \end{beamercolorbox}
}
\setbeamertemplate{subsection page}{
  \centering
  \begin{beamercolorbox}[sep=8pt,center]{part title}
    \usebeamerfont{subsection title}\insertsubsection\par
  \end{beamercolorbox}
}
\AtBeginPart{
  \frame{\partpage}
}
\AtBeginSection{
  \ifbibliography
  \else
    \frame{\sectionpage}
  \fi
}
\AtBeginSubsection{
  \frame{\subsectionpage}
}
\usepackage{lmodern}
\usepackage{amssymb,amsmath}
\usepackage{ifxetex,ifluatex}
\ifnum 0\ifxetex 1\fi\ifluatex 1\fi=0 % if pdftex
  \usepackage[T1]{fontenc}
  \usepackage[utf8]{inputenc}
  \usepackage{textcomp} % provide euro and other symbols
\else % if luatex or xetex
  \usepackage{unicode-math}
  \defaultfontfeatures{Scale=MatchLowercase}
  \defaultfontfeatures[\rmfamily]{Ligatures=TeX,Scale=1}
\fi
% Use upquote if available, for straight quotes in verbatim environments
\IfFileExists{upquote.sty}{\usepackage{upquote}}{}
\IfFileExists{microtype.sty}{% use microtype if available
  \usepackage[]{microtype}
  \UseMicrotypeSet[protrusion]{basicmath} % disable protrusion for tt fonts
}{}
\makeatletter
\@ifundefined{KOMAClassName}{% if non-KOMA class
  \IfFileExists{parskip.sty}{%
    \usepackage{parskip}
  }{% else
    \setlength{\parindent}{0pt}
    \setlength{\parskip}{6pt plus 2pt minus 1pt}}
}{% if KOMA class
  \KOMAoptions{parskip=half}}
\makeatother
\usepackage{xcolor}
\IfFileExists{xurl.sty}{\usepackage{xurl}}{} % add URL line breaks if available
\IfFileExists{bookmark.sty}{\usepackage{bookmark}}{\usepackage{hyperref}}
\hypersetup{
  pdftitle={165\_Markdown},
  pdfauthor={achiola},
  hidelinks,
  pdfcreator={LaTeX via pandoc}}
\urlstyle{same} % disable monospaced font for URLs
\newif\ifbibliography
\usepackage{color}
\usepackage{fancyvrb}
\newcommand{\VerbBar}{|}
\newcommand{\VERB}{\Verb[commandchars=\\\{\}]}
\DefineVerbatimEnvironment{Highlighting}{Verbatim}{commandchars=\\\{\}}
% Add ',fontsize=\small' for more characters per line
\usepackage{framed}
\definecolor{shadecolor}{RGB}{248,248,248}
\newenvironment{Shaded}{\begin{snugshade}}{\end{snugshade}}
\newcommand{\AlertTok}[1]{\textcolor[rgb]{0.94,0.16,0.16}{#1}}
\newcommand{\AnnotationTok}[1]{\textcolor[rgb]{0.56,0.35,0.01}{\textbf{\textit{#1}}}}
\newcommand{\AttributeTok}[1]{\textcolor[rgb]{0.77,0.63,0.00}{#1}}
\newcommand{\BaseNTok}[1]{\textcolor[rgb]{0.00,0.00,0.81}{#1}}
\newcommand{\BuiltInTok}[1]{#1}
\newcommand{\CharTok}[1]{\textcolor[rgb]{0.31,0.60,0.02}{#1}}
\newcommand{\CommentTok}[1]{\textcolor[rgb]{0.56,0.35,0.01}{\textit{#1}}}
\newcommand{\CommentVarTok}[1]{\textcolor[rgb]{0.56,0.35,0.01}{\textbf{\textit{#1}}}}
\newcommand{\ConstantTok}[1]{\textcolor[rgb]{0.00,0.00,0.00}{#1}}
\newcommand{\ControlFlowTok}[1]{\textcolor[rgb]{0.13,0.29,0.53}{\textbf{#1}}}
\newcommand{\DataTypeTok}[1]{\textcolor[rgb]{0.13,0.29,0.53}{#1}}
\newcommand{\DecValTok}[1]{\textcolor[rgb]{0.00,0.00,0.81}{#1}}
\newcommand{\DocumentationTok}[1]{\textcolor[rgb]{0.56,0.35,0.01}{\textbf{\textit{#1}}}}
\newcommand{\ErrorTok}[1]{\textcolor[rgb]{0.64,0.00,0.00}{\textbf{#1}}}
\newcommand{\ExtensionTok}[1]{#1}
\newcommand{\FloatTok}[1]{\textcolor[rgb]{0.00,0.00,0.81}{#1}}
\newcommand{\FunctionTok}[1]{\textcolor[rgb]{0.00,0.00,0.00}{#1}}
\newcommand{\ImportTok}[1]{#1}
\newcommand{\InformationTok}[1]{\textcolor[rgb]{0.56,0.35,0.01}{\textbf{\textit{#1}}}}
\newcommand{\KeywordTok}[1]{\textcolor[rgb]{0.13,0.29,0.53}{\textbf{#1}}}
\newcommand{\NormalTok}[1]{#1}
\newcommand{\OperatorTok}[1]{\textcolor[rgb]{0.81,0.36,0.00}{\textbf{#1}}}
\newcommand{\OtherTok}[1]{\textcolor[rgb]{0.56,0.35,0.01}{#1}}
\newcommand{\PreprocessorTok}[1]{\textcolor[rgb]{0.56,0.35,0.01}{\textit{#1}}}
\newcommand{\RegionMarkerTok}[1]{#1}
\newcommand{\SpecialCharTok}[1]{\textcolor[rgb]{0.00,0.00,0.00}{#1}}
\newcommand{\SpecialStringTok}[1]{\textcolor[rgb]{0.31,0.60,0.02}{#1}}
\newcommand{\StringTok}[1]{\textcolor[rgb]{0.31,0.60,0.02}{#1}}
\newcommand{\VariableTok}[1]{\textcolor[rgb]{0.00,0.00,0.00}{#1}}
\newcommand{\VerbatimStringTok}[1]{\textcolor[rgb]{0.31,0.60,0.02}{#1}}
\newcommand{\WarningTok}[1]{\textcolor[rgb]{0.56,0.35,0.01}{\textbf{\textit{#1}}}}
\usepackage{graphicx,grffile}
\makeatletter
\def\maxwidth{\ifdim\Gin@nat@width>\linewidth\linewidth\else\Gin@nat@width\fi}
\def\maxheight{\ifdim\Gin@nat@height>\textheight\textheight\else\Gin@nat@height\fi}
\makeatother
% Scale images if necessary, so that they will not overflow the page
% margins by default, and it is still possible to overwrite the defaults
% using explicit options in \includegraphics[width, height, ...]{}
\setkeys{Gin}{width=\maxwidth,height=\maxheight,keepaspectratio}
% Set default figure placement to htbp
\makeatletter
\def\fps@figure{htbp}
\makeatother
\setlength{\emergencystretch}{3em} % prevent overfull lines
\providecommand{\tightlist}{%
  \setlength{\itemsep}{0pt}\setlength{\parskip}{0pt}}
\setcounter{secnumdepth}{-\maxdimen} % remove section numbering

\title{165\_Markdown}
\author{achiola}
\date{2/1/2020}

\begin{document}
\frame{\titlepage}

\hypertarget{principiantes}{%
\section{Principiantes}\label{principiantes}}

\begin{frame}

Delimitado por linea

\textbf{negrita}

En cursiva con html

\end{frame}

\begin{frame}

\#HTML contents

Esto es un parrafo en HTML

Ventajas

Inconvenientes

V1

I1

V2

I2

V3

I3

\end{frame}

\hypertarget{embed-code}{%
\section{Embed Code}\label{embed-code}}

\begin{frame}{Set working directory}
\protect\hypertarget{set-working-directory}{}

\begin{itemize}
\tightlist
\item
  Se puede definir cualquier codigo en \emph{R} haciendo uso de los tre
  comillas simples.
\item
  El parametro echo=FALSE, indica que el codigo no se mostra en el
  documento.
\item
  Se puede establecer el directorio de trabajo, estableciendo root.dir
\end{itemize}

\end{frame}

\begin{frame}[fragile]{Loading data}
\protect\hypertarget{loading-data}{}

\begin{block}{Muestra de los datos (n primeros)}

\begin{Shaded}
\begin{Highlighting}[]
  \KeywordTok{head}\NormalTok{(auto)}
\end{Highlighting}
\end{Shaded}

\begin{verbatim}
##   No mpg cylinders displacement horsepower weight acceleration model_year            car_name
## 1  1  28         4          140         90   2264         15.5         71 chevrolet vega 2300
## 2  2  19         3           70         97   2330         13.5         72     mazda rx2 coupe
## 3  3  36         4          107         75   2205         14.5         82        honda accord
## 4  4  28         4           97         92   2288         17.0         72     datsun 510 (sw)
## 5  5  21         6          199         90   2648         15.0         70         amc gremlin
## 6  6  23         4          115         95   2694         15.0         75          audi 100ls
\end{verbatim}

\end{block}

\begin{block}{Estructura del df}

\begin{Shaded}
\begin{Highlighting}[]
  \KeywordTok{str}\NormalTok{(auto)}
\end{Highlighting}
\end{Shaded}

\begin{verbatim}
## 'data.frame':    398 obs. of  9 variables:
##  $ No          : int  1 2 3 4 5 6 7 8 9 10 ...
##  $ mpg         : num  28 19 36 28 21 23 15.5 32.9 16 13 ...
##  $ cylinders   : int  4 3 4 4 6 4 8 4 6 8 ...
##  $ displacement: num  140 70 107 97 199 115 304 119 250 318 ...
##  $ horsepower  : int  90 97 75 92 90 95 120 100 105 150 ...
##  $ weight      : int  2264 2330 2205 2288 2648 2694 3962 2615 3897 3755 ...
##  $ acceleration: num  15.5 13.5 14.5 17 15 15 13.9 14.8 18.5 14 ...
##  $ model_year  : int  71 72 82 72 70 75 76 81 75 76 ...
##  $ car_name    : Factor w/ 305 levels "amc ambassador brougham",..: 66 184 165 86 8 18 11 79 42 112 ...
\end{verbatim}

\end{block}

\begin{block}{resumen (summary)}

\begin{Shaded}
\begin{Highlighting}[]
  \KeywordTok{summary}\NormalTok{(auto)}
\end{Highlighting}
\end{Shaded}

\begin{verbatim}
##        No             mpg          cylinders      displacement     horsepower        weight    
##  Min.   :  1.0   Min.   : 9.00   Min.   :3.000   Min.   : 68.0   Min.   : 46.0   Min.   :1613  
##  1st Qu.:100.2   1st Qu.:17.50   1st Qu.:4.000   1st Qu.:104.2   1st Qu.: 76.0   1st Qu.:2224  
##  Median :199.5   Median :23.00   Median :4.000   Median :148.5   Median : 92.0   Median :2804  
##  Mean   :199.5   Mean   :23.51   Mean   :5.455   Mean   :193.4   Mean   :104.1   Mean   :2970  
##  3rd Qu.:298.8   3rd Qu.:29.00   3rd Qu.:8.000   3rd Qu.:262.0   3rd Qu.:125.0   3rd Qu.:3608  
##  Max.   :398.0   Max.   :46.60   Max.   :8.000   Max.   :455.0   Max.   :230.0   Max.   :5140  
##                                                                                                
##   acceleration     model_year              car_name  
##  Min.   : 8.00   Min.   :70.00   ford pinto    :  6  
##  1st Qu.:13.82   1st Qu.:73.00   amc matador   :  5  
##  Median :15.50   Median :76.00   ford maverick :  5  
##  Mean   :15.57   Mean   :76.01   toyota corolla:  5  
##  3rd Qu.:17.18   3rd Qu.:79.00   amc gremlin   :  4  
##  Max.   :24.80   Max.   :82.00   amc hornet    :  4  
##                                  (Other)       :369
\end{verbatim}

\end{block}

\begin{block}{Plot}

\includegraphics{165_Markdown_files/figure-beamer/disperssion-1.pdf}

\end{block}

\end{frame}

\begin{frame}{R inline}
\protect\hypertarget{r-inline}{}

Hemos hecho uso de un data frame de coches, que contiene \textbf{398}
muestras de coches y cada uno de ellos tiene \textbf{9} variables.

Fin

\end{frame}

\end{document}
